\documentclass{article}

\usepackage{amsmath}
\usepackage{amssymb}
\usepackage{graphicx}

\begin{document}
\author{\textit{Xetera Mnemonics}\\\small{grostaco@gmail.com}}
\title{Sequences and Limits}
\maketitle

\section{The sequence $(x_n)$ is defined by the following formulas for the \textit{n}th term. Write the first five terms in each case}
\subsection{$x_n := 1 + (-1)^n$}

\begin{align*}
    x_1 & = 0 \\
    x_2 & = 1 \\
    x_3 & = 0 \\
    x_4 & = 1 \\
    x_5 & = 0
\end{align*}

\subsection{$x_n := \frac{(-1)^n}{n}$}

\begin{align*}
    x_1 & = -1 \\
    x_2 & = \frac{1}{2} \\
    x_3 & = \frac{-1}{3} \\
    x_4 & = \frac{-1}{4} \\
    x_5 & = \frac{1}{5}  
\end{align*}

\subsection{$x_n := \frac{1}{n(n + 1)}$}
\begin{align*}
    x_1 & = \frac{1}{2} \\
    x_2 & = \frac{1}{6} \\
    x_3 & = \frac{1}{12} \\
    x_4 & = \frac{1}{20} \\
    x_5 & = \frac{1}{30}  
\end{align*}

\subsection{$x_n := \frac{1}{n^2 + 2}$}
\begin{align*}
    x_1 & = \frac{1}{3} \\
    x_2 & = \frac{1}{6} \\
    x_3 & = \frac{1}{11} \\
    x_4 & = \frac{1}{18} \\
    x_5 & = \frac{1}{27}  
\end{align*}

\section{The first few terms of a sequence $(x_n)$ are given below. Assuming that the "natural pattern" indicated by these term persists, give a formula for the \textit{n}th term $x_n$}

\subsection{$5$, $7$, $8$, $11$, ...,}
\begin{equation*}
    x_n = 3 + 2n
\end{equation*}

\subsection{$\frac{1}{2}$, $-\frac{1}{4}$, $\frac{1}{8}$, $-\frac{1}{16}$, ...,}
\begin{equation*}
    x_n = -(-\frac{1}{2})^n
\end{equation*}

\subsection{$\frac{1}{2}$, $-\frac{2}{3}$, $\frac{3}{4}$, $-\frac{4}{5}$, ...,}
\begin{equation*}
    x_n = \frac{n}{n+1}
\end{equation*}

\subsection{1, 4, 9, 16, ...,}
\begin{equation*}
    x_n = n^2
\end{equation*}

\section{List the first five terms of the following inductively defined sequences}
\subsection{$x_1 := 1$, $x_{n + 1} := 3x_n + 1$}

\begin{align*}
    x_2 & = 4 \\
    x_3 & = 13 \\
    x_4 & = 40 \\
    x_5 & = 121 \\
    x_6 & = 364 
\end{align*}

\section{For any $b \in \mathbb{R}$, prove that $\lim(\frac{b}{n}) = 0$}
If $\epsilon > 0$ is given, then $\frac{1}{\epsilon} > 0$. By the Archimedean property, there exists a natural number $K = K(\epsilon)$ such that $\frac{b}{K} < \epsilon$.
Then, if $n \ge K$, we have that $\frac{b}{n} \le \frac{b}{K} < \epsilon$. Consequently, if $n \ge K$, then 

\begin{equation*}
    \Big|\frac{b}{n} - 0 \Big| = \frac{b}{n} < \epsilon 
\end{equation*}
$\therefore \lim(\frac{b}{n}) = 0$ 

\section{Use the definition of the limit of a sequence to establish the following limits}
\subsection{$\lim(\frac{n}{n^2 + 1}) = 0$}
Let $\epsilon > 0$ be given. Let us first note that $n \in \mathbb{N}$ and
\begin{equation*}
    \frac{n}{n^2 + 1} < \frac{n}{n^2} = \frac{1}{n}
\end{equation*}
From the Archimedean Property there exists $K \in \mathbb{N}$ such that, $\frac{1}{K} < \epsilon$.
If $n \ge K$, then $\frac{1}{n} \le \frac{1}{K} < \epsilon$, therefore 

\begin{equation*}
    \Big| \frac{n}{n^2 + 1} - 0 \Big| = \frac{n}{n^2 + 1} < \frac{n}{n^2} = \frac{1}{n} < \epsilon
\end{equation*}
$\therefore \lim(\frac{n}{n^2 + 1}) = 0$

\subsection{$\lim(\frac{2n}{n + 1}) = 2$}
Let $\epsilon > 0$ be given, we want to obtain the inequality 

\begin{equation*}
    \Big| \frac{2n}{n + 1} - 2 \Big| < \epsilon
\end{equation*}
when $n \ge K$ for some $K \in \mathbb{N}$. We can simplify the expression on the left:
\begin{align*}
    \Big| \frac{2n}{n + 1} - 2 \Big| & = \Big| \frac{2n - 2n - 2}{n + 1} \Big| \\
                                     & = \Big| \frac{-2}{n + 1} \Big| \\
                                    & = \frac{2}{n + 1} < \frac{2}{n} < \epsilon
\end{align*}
From section 4, it is clear that since $\forall b \in R \Rightarrow \lim(\frac{b}{n}) = 0$, then $| \frac{2}{n} - 0 | = \frac{2}{n} < \epsilon$\\


$\therefore \lim(\frac{2n}{n^2 + 1}) = 2$


\subsection{$\lim(\frac{3n + 1}{2n + 5}) = \frac{3}{2}$}
Let $\epsilon > 0$ be given, we want to obtain the inequality 

\begin{equation*}
    \Big| \frac{3n + 1}{2n + 5} - \frac{3}{2} \Big| < \epsilon
\end{equation*}
when $n \ge K$ for some $K \in \mathbb{N}$. We can simplify the expression on the left:
\begin{align*}
    \Big| \frac{3n + 1}{2n + 5} - \frac{3}{2} \Big| & = \Big| \frac{6n + 2 - 6n - 15}{4n + 5} \Big| \\
                                     & = \Big| \frac{-13}{4n + 5} \Big| \\
                                    & = \frac{13}{4n + 5} < \frac{13}{n} = \frac{13}{n} < \epsilon
\end{align*}
From section 4, it is clear that since $\forall b \in R \Rightarrow \lim(\frac{b}{n}) = 0$, then $| \frac{13}{n} - 0 | = \frac{13}{n} < \epsilon$ \\
$\therefore \lim(\frac{3n + 1}{2n + 5}) = \frac{3}{2}$

\subsection{$\lim(\frac{n^2 - 1}{2n^2 + 3}) = \frac{1}{2}$}
Let $\epsilon > 0$ be given, we want to obtain the inequality 

\begin{equation*}
    \Big| \frac{n^2 - 1}{2n^2 + 3} - \frac{1}{2} \Big| < \epsilon
\end{equation*}
when $n \ge K$ for some $K \in \mathbb{N}$. We can simplify the expression on the left:
\begin{align*}
    \Big| \frac{n^2 - 1}{2n^2 + 3} - \frac{1}{2} \Big| & = \Big| \frac{2n^2 - 2 - 2n^2 - 3}{4n^2 + 6} \Big| \\
                                     & = \Big| \frac{-5}{4n^2 + 6} \Big| \\
                                    & = \frac{5}{4n^2 + 6} < \frac{5}{4n^2} = \frac{5}{n^2} \le \frac{5}{n} < \epsilon
\end{align*}
From section 4, it is clear that since $\forall b \in R \Rightarrow \lim(\frac{b}{n}) = 0$, then $| \frac{5}{n} - 0 | = \frac{5}{n} < \epsilon$ \\
$\therefore \lim(\frac{3n + 1}{2n + 5}) = \frac{3}{2}$

\section{Show that}
\subsection{$\lim(\frac{1}{\sqrt{n + 7}}) = 0$}

We should first note that since $n \in \mathbb{N}$,
\begin{equation*}
    \frac{1}{\sqrt{n + 7}} < \frac{1}{\sqrt{n}}
\end{equation*}

For a given $\epsilon > 0$, we obtain $1/\sqrt{n} < \epsilon$ iff $n > 1/\epsilon^2$. 
If we take $K > 1/\epsilon^2$ then it follows that for all $n \ge K$:

\begin{equation*}
    \Big|\frac{1}{\sqrt{n}} - 0 \Big| < \epsilon
\end{equation*}

\subsection{$\lim(\frac{2n}{n + 2}) = 2$}
We should first note that since $n \in \mathbb{N}$,
\begin{equation*}
    \frac{2n}{\sqrt{n + 2}} < \frac{2n}{n} = 2
\end{equation*}
It is trivial to show that $|2 - 2| < \epsilon$ for all $\epsilon > 0$


\subsection{$\lim(\frac{\sqrt{n}}{n + 1}) = 0$}
We should first note that since $n \in \mathbb{N}$,
\begin{equation*}
    \frac{\sqrt{n}}{n + 1} < \frac{\sqrt{n}}{n} = \frac{1}{\sqrt{n}}
\end{equation*}
It is sufficient from 6.1 to conclude that $\lim(\frac{\sqrt{n}}{n + 1}) = 0$

\subsection{$\lim(\frac{(-1)^n \cdot n}{n^2 + 1}) = 0$}
We should first note that since $n \in \mathbb{N}$,
\begin{equation*}
    \frac{(-1)^n \cdot n}{n^2 + 1} < \frac{n}{n^2 + 1} < \frac{n}{n^2} = \frac{1}{n}
\end{equation*}
The result follows from section 4

\end{document}